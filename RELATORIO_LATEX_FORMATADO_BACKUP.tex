\section{Sumário Executivo}

Este projeto implementa um \textbf{Sistema de Recomendação de Animes} utilizando técnicas de \textbf{Filtragem Colaborativa} baseada em usuários. O sistema foi desenvolvido seguindo a metodologia \textbf{CRISP-DM} (Cross-Industry Standard Process for Data Mining), abrangendo todas as etapas desde o entendimento do negócio até a avaliação do modelo.

\subsection{Principais Resultados}

\begin{itemize}
    \item \textbf{MAE (Mean Absolute Error):} 0.7682 - Excelente resultado (meta: < 1.5)
    \item \textbf{RMSE (Root Mean Squared Error):} 1.0210 - Baixo erro nas predições
    \item \textbf{Cobertura:} 5.000 usuários mais ativos do dataset
    \item \textbf{Dataset:} 24.905 animes e milhões de avaliações do MyAnimeList
\end{itemize}

\section{Entendimento do Negócio (Business Understanding)}

\subsection{Problema de Negócio}

Plataformas de streaming e comunidades de animes enfrentam desafios significativos:

\textbf{Desafios Identificados:}
\begin{itemize}
    \item \textbf{Catálogo extenso:} Mais de 24.000 animes disponíveis
    \item \textbf{Dificuldade de descoberta:} Usuários gastam muito tempo procurando novos conteúdos
    \item \textbf{Baixo engajamento:} Usuários abandonam a plataforma por não encontrar conteúdo relevante
    \item \textbf{Impacto financeiro:} Redução de retenção e satisfação dos usuários
\end{itemize}

\subsection{Objetivo do Projeto}

Desenvolver um \textbf{Sistema de Recomendação Inteligente} que:

\begin{enumerate}
    \item \textbf{Personalize sugestões} baseadas no histórico e preferências do usuário
    \item \textbf{Aumente o engajamento} reduzindo o tempo de busca
    \item \textbf{Melhore a satisfação} sugerindo conteúdos alinhados com os gostos individuais
    \item \textbf{Utilize padrões colaborativos} identificando usuários com gostos similares
\end{enumerate}

\subsection{Métricas de Sucesso (KPIs)}

\begin{table}[h]
\centering
\begin{tabular}{|l|c|c|c|}
\hline
\textbf{Métrica} & \textbf{Meta} & \textbf{Resultado Obtido} & \textbf{Status} \\ \hline
MAE & < 1.5 & 0.7682 & Superado \\ \hline
RMSE & < 2.0 & 1.0210 & Superado \\ \hline
Cobertura & > 1000 usuários & 5.000 usuários & Superado \\ \hline
\end{tabular}
\caption{Métricas de Sucesso do Projeto}
\end{table}

\textbf{Interpretação:}
\begin{itemize}
    \item Em média, as predições erram por \textbf{0.77 pontos} em uma escala de 1-10
    \item O modelo demonstra \textbf{alta precisão} nas recomendações
    \item Capacidade de atender \textbf{milhares de usuários} simultaneamente
\end{itemize}

\section{Entendimento dos Dados (Data Understanding)}

\subsection{Fonte dos Dados}

\textbf{Dataset:} MyAnimeList Dataset 2023 (Kaggle)

\textbf{Arquivos Utilizados:}
\begin{enumerate}
    \item \texttt{anime-dataset-2023.csv} - Metadados dos animes (24.905 registros)
    \item \texttt{user-dataset-2023.csv} - Avaliações dos usuários (milhões de registros)
\end{enumerate}

\subsection{Estrutura dos Dados}

\subsubsection{DataFrame de Animes (df1)}
\textbf{Dimensões:} 24.905 animes × 24 colunas

\textbf{Principais Colunas:}
\begin{itemize}
    \item \texttt{anime\_id}: Identificador único do anime
    \item \texttt{Name}: Nome do anime
    \item \texttt{English name}: Nome em inglês
    \item \texttt{Score}: Pontuação média no MyAnimeList
    \item \texttt{Genres}: Gêneros (Action, Comedy, Drama, etc.)
    \item \texttt{Type}: Tipo (TV, Movie, OVA, Special, etc.)
    \item \texttt{Episodes}: Número de episódios
    \item \texttt{Studios}: Estúdio de produção
    \item \texttt{Members}: Número de membros que adicionaram à lista
    \item \texttt{Favorites}: Número de usuários que favoritaram
\end{itemize}

\subsubsection{DataFrame de Usuários (df2)}
\textbf{Dimensões:} Milhões de avaliações × 5 colunas

\textbf{Principais Colunas:}
\begin{itemize}
    \item \texttt{user\_id}: Identificador único do usuário
    \item \texttt{Username}: Nome do usuário
    \item \texttt{anime\_id}: ID do anime avaliado
    \item \texttt{Anime Title}: Título do anime
    \item \texttt{rating}: Avaliação do usuário (escala 1-10)
\end{itemize}

\subsection{Análise Exploratória de Dados (EDA)}

\subsubsection{Estatísticas Descritivas}

\textbf{Dataset de Animes:}
\begin{itemize}
    \item Total de animes: 24.905
    \item Score médio: $\sim$7.0
    \item Distribuição de tipos: TV (maioria), Movies, OVAs, Specials
    \item Gêneros mais comuns: Action, Comedy, Drama, Fantasy
\end{itemize}

\textbf{Dataset de Usuários:}
\begin{itemize}
    \item Milhões de avaliações
    \item Escala de ratings: 1 a 10
    \item Distribuição: Viés positivo (usuários tendem a avaliar melhor o que escolhem assistir)
\end{itemize}

\subsubsection{Insights Principais}

\textbf{1. Esparsidade da Matriz}
\begin{itemize}
    \item Densidade < 1\%
    \item Maioria dos usuários avalia poucos animes
    \item Desafio: Matriz extremamente esparsa
\end{itemize}

\textbf{2. Distribuição de Ratings}
\begin{itemize}
    \item Ratings mais comuns: 7, 8, 9, 10
    \item Viés positivo: Usuários avaliam melhor animes que escolhem
    \item Poucos ratings baixos (1-4)
\end{itemize}

\textbf{3. Popularidade dos Animes}
\begin{itemize}
    \item Distribuição long tail
    \item Poucos animes (One Piece, Naruto, Attack on Titan) concentram muitas avaliações
    \item Maioria dos animes tem poucas avaliações
\end{itemize}

\textbf{4. Padrões de Usuários}
\begin{itemize}
    \item Usuários ativos avaliam centenas de animes
    \item Usuários casuais avaliam poucos (< 10)
    \item Oportunidade: Focar em usuários mais ativos
\end{itemize}

\subsubsection{Desafios Identificados}

\begin{itemize}
    \item \textbf{Cold Start Problem:} Dificuldade em recomendar para novos usuários/animes
    \item \textbf{Esparsidade:} Poucos dados para alguns pares usuário-anime
    \item \textbf{Escalabilidade:} Milhões de combinações possíveis
    \item \textbf{Viés de Popularidade:} Tendência a recomendar animes populares
\end{itemize}

\section{Preparação dos Dados (Data Preparation)}

\subsection{Estratégia de Preparação}

Implementamos um \textbf{pipeline completo} usando Scikit-learn com transformadores customizados, garantindo:
\begin{itemize}
    \item \textbf{Reprodutibilidade:} Mesmos resultados em execuções diferentes
    \item \textbf{Modularidade:} Cada etapa é independente e pode ser modificada
    \item \textbf{Manutenibilidade:} Código mais limpo e fácil de entender
    \item \textbf{Prevenção de Data Leakage:} Transformações aplicadas corretamente
\end{itemize}

\subsection{Transformadores Customizados}

Criamos transformadores customizados que herdam de \texttt{BaseEstimator} e \texttt{TransformerMixin}:

\subsubsection{1. UserAnimeIndexMapper}
\textbf{Função:} Mapeia IDs de usuários e animes para índices numéricos
\begin{itemize}
    \item Entrada: \texttt{user\_id}, \texttt{anime\_id}
    \item Saída: \texttt{user\_idx}, \texttt{anime\_idx}
    \item Benefício: Facilita criação de matrizes esparsas
\end{itemize}

\subsubsection{2. SparseMatrixCreator}
\textbf{Função:} Cria matriz esparsa de interações usuário-item
\begin{itemize}
    \item Formato: CSR (Compressed Sparse Row)
    \item Dimensões: (n\_users, n\_animes)
    \item Valores: ratings dos usuários
    \item Benefício: Eficiência de memória (densidade < 1\%)
\end{itemize}

\subsubsection{3. TopUserSelector}
\textbf{Função:} Seleciona os usuários mais ativos
\begin{itemize}
    \item Critério: Número de avaliações
    \item Amostra: 5.000 usuários mais ativos
    \item Benefício: Viabiliza processamento em tempo razoável
\end{itemize}

\subsubsection{4. RatingCentralizer}
\textbf{Função:} Centraliza ratings pela média de cada usuário
\begin{itemize}
    \item Fórmula: \texttt{centered\_rating = rating - user\_mean}
    \item Benefício: Remove viés individual de cada usuário
    \item Melhora: Similaridade entre usuários mais precisa
\end{itemize}

\subsubsection{5. SimilarityMatrixCalculator}
\textbf{Função:} Calcula matriz de similaridade entre usuários
\begin{itemize}
    \item Método: Similaridade de Cosseno
    \item Dimensões: (n\_users, n\_users)
    \item Benefício: Identifica usuários com gostos similares
\end{itemize}

\subsection{Processo de Preparação}

\textbf{Etapa 1: Seleção de Amostra}
\begin{itemize}
    \item Dataset original $\rightarrow$ 5.000 usuários mais ativos
    \item Critério: Número de avaliações por usuário
    \item Resultado: Dataset reduzido mas representativo
\end{itemize}

\textbf{Etapa 2: Mapeamento de Índices}
\begin{itemize}
    \item \texttt{user\_id} $\rightarrow$ \texttt{user\_idx} (0 a 4.999)
    \item \texttt{anime\_id} $\rightarrow$ \texttt{anime\_idx} (0 a n\_animes-1)
\end{itemize}

\textbf{Etapa 3: Criação de Matriz Esparsa}
\begin{itemize}
    \item Formato: CSR Matrix
    \item Shape: (5.000 usuários, 24.905 animes)
    \item Densidade: < 1\%
    \item Tamanho em memória: Otimizado
\end{itemize}

\textbf{Etapa 4: Centralização de Ratings}
\begin{itemize}
    \item Para cada usuário:
    \begin{itemize}
        \item Calcula média de ratings
        \item Subtrai média de cada rating
        \item Resultado: ratings centralizados
    \end{itemize}
\end{itemize}

\textbf{Etapa 5: Cálculo de Similaridade}
\begin{itemize}
    \item Método: Cosine Similarity
    \item Entrada: Matriz usuário-item centralizada
    \item Saída: Matriz de similaridade (5.000 × 5.000)
    \item Tempo de processamento: $\sim$1-2 minutos
\end{itemize}

\subsection{Decisões de Design}

\textbf{Por que amostragem de 5.000 usuários?}
\begin{itemize}
    \item Balanceamento entre qualidade e performance
    \item Tempo de processamento aceitável ($\sim$2 minutos)
    \item Memória RAM suficiente para matriz de similaridade
    \item Usuários mais ativos = mais dados para recomendações
\end{itemize}

\textbf{Por que centralização de ratings?}
\begin{itemize}
    \item Remove viés individual (alguns usuários avaliam sempre alto/baixo)
    \item Melhora cálculo de similaridade
    \item Foca em preferências relativas, não absolutas
\end{itemize}

\textbf{Por que matriz esparsa?}
\begin{itemize}
    \item Economia de memória (99\% dos valores são zero)
    \item Operações otimizadas com \texttt{scipy.sparse}
    \item Viabiliza processamento de grandes volumes
\end{itemize}

\textbf{Por que similaridade de cosseno?}
\begin{itemize}
    \item Métrica padrão para sistemas de recomendação
    \item Invariante à magnitude (foca em padrões, não em valores absolutos)
    \item Eficiente computacionalmente
    \item Valores entre -1 e 1 (fácil interpretação)
\end{itemize}

\section{Modelagem (Modeling)}

\subsection{Algoritmo Escolhido}

\textbf{Filtragem Colaborativa Baseada em Usuários (User-Based Collaborative Filtering)}

\textbf{Princípio:} ``Usuários com histórico de avaliações similares tendem a ter gostos parecidos''

\textbf{Funcionamento:}
\begin{enumerate}
    \item Identifica usuários similares ao usuário alvo
    \item Analisa o que esses usuários similares gostaram
    \item Recomenda itens bem avaliados por usuários similares
\end{enumerate}

\subsection{Justificativa da Escolha}

\textbf{Por que Filtragem Colaborativa?}

\begin{enumerate}
    \item \textbf{Interpretabilidade}
    \begin{itemize}
        \item Fácil explicar: ``Usuários com gostos similares aos seus gostaram deste anime''
        \item Transparência nas recomendações
    \end{itemize}

    \item \textbf{Eficácia Comprovada}
    \begin{itemize}
        \item Amplamente usado na indústria (Netflix, Amazon, Spotify)
        \item Resultados consistentes em diversos domínios
    \end{itemize}

    \item \textbf{Adequação ao Dataset}
    \begin{itemize}
        \item Dataset rico em avaliações de usuários
        \item Padrões claros de preferências
        \item Densidade suficiente para usuários ativos
    \end{itemize}

    \item \textbf{Implementação Viável}
    \begin{itemize}
        \item Não requer infraestrutura complexa
        \item Processamento em tempo razoável
        \item Não necessita GPU
    \end{itemize}
\end{enumerate}

\textbf{Alternativas Consideradas:}

\begin{table}[h]
\centering
\small
\begin{tabular}{|p{3cm}|p{4cm}|p{4cm}|p{2cm}|}
\hline
\textbf{Algoritmo} & \textbf{Vantagens} & \textbf{Desvantagens} & \textbf{Decisão} \\ \hline
Item-Based CF & Mais escalável & Menos personalizado & Futuro \\ \hline
Matrix Factorization & Lida melhor com esparsidade & Menos interpretável & Futuro \\ \hline
Deep Learning & Alta precisão potencial & Requer mais dados/recursos & Futuro \\ \hline
Content-Based & Resolve cold start & Ignora padrões colaborativos & Híbrido futuro \\ \hline
\end{tabular}
\caption{Comparação de Algoritmos de Recomendação}
\end{table}

\subsection{Implementação do Algoritmo}

\subsubsection{Função de Predição de Rating}

\textbf{Assinatura:}
\begin{verbatim}
def predict_rating(user_idx, anime_idx, k=50):
    """
    Prediz o rating que um usuário daria para um anime

    Parâmetros:
    - user_idx: Índice do usuário
    - anime_idx: Índice do anime
    - k: Número de vizinhos similares a considerar

    Retorna:
    - rating_predito: Float entre 1 e 10
    """
\end{verbatim}

\textbf{Algoritmo:}
\begin{enumerate}
    \item Obtém vetor de similaridade do usuário alvo
    \item Identifica top-k usuários mais similares
    \item Filtra usuários que avaliaram o anime alvo
    \item Calcula média ponderada dos ratings
    \item Ajusta pela média do usuário
    \item Retorna predição
\end{enumerate}

\textbf{Fórmula Matemática:}
\begin{equation}
pred(u, i) = \overline{r}_u + \frac{\sum_{v \in N_k(u)} sim(u, v) \cdot (r_{v,i} - \overline{r}_v)}{\sum_{v \in N_k(u)} |sim(u, v)|}
\end{equation}

Onde:
\begin{itemize}
    \item $u$ = usuário alvo
    \item $i$ = anime alvo
    \item $v$ = usuário vizinho
    \item $sim(u, v)$ = similaridade entre $u$ e $v$
    \item $r_{v,i}$ = rating do usuário $v$ para anime $i$
    \item $\overline{r}_u$ = média de ratings do usuário $u$
    \item $N_k(u)$ = conjunto dos $k$ vizinhos mais similares a $u$
\end{itemize}

\subsubsection{Passos do Algoritmo}

\textbf{Passo 1: Cálculo de Similaridade}
\begin{itemize}
    \item Método: Similaridade de Cosseno
    \item Entrada: Matriz usuário-item centralizada
    \item Saída: Matriz de similaridade (n\_users × n\_users)
    \item Valores: -1 (opostos) a +1 (idênticos)
\end{itemize}

\textbf{Passo 2: Identificação de Vizinhos}
\begin{itemize}
    \item Para cada predição, seleciona top-k usuários mais similares
    \item Filtra apenas usuários que avaliaram o anime alvo
    \item $k = 50$ (hiperparâmetro)
\end{itemize}

\textbf{Passo 3: Predição Ponderada}
\begin{itemize}
    \item Calcula média ponderada dos ratings dos vizinhos
    \item Peso = similaridade entre usuários
    \item Ajusta pela média do usuário alvo
    \item Fórmula: \texttt{user\_mean + weighted\_average(neighbor\_ratings\_centered)}
\end{itemize}

\textbf{Passo 4: Geração de Recomendações}

Para usuário $u$:
\begin{enumerate}
    \item Identifica animes NÃO assistidos
    \item Prediz rating para cada anime
    \item Ordena por rating predito (decrescente)
    \item Retorna top-N recomendações
\end{enumerate}

\subsection{Hiperparâmetros}

\begin{table}[h]
\centering
\begin{tabular}{|l|c|p{7cm}|}
\hline
\textbf{Parâmetro} & \textbf{Valor} & \textbf{Descrição} \\ \hline
k\_neighbors & 50 & Número de vizinhos similares considerados \\ \hline
n\_users\_sample & 5.000 & Usuários mais ativos na amostra \\ \hline
n\_recommendations & 10 & Número de recomendações retornadas \\ \hline
\end{tabular}
\caption{Hiperparâmetros do Modelo}
\end{table}

\textbf{Justificativa k=50:}
\begin{itemize}
    \item Balanceamento entre diversidade e precisão
    \item Suficiente para capturar padrões robustos
    \item Evita overfitting de poucos usuários
    \item Testado empiricamente com bons resultados
\end{itemize}

\subsection{Exemplo de Recomendação}

\textbf{Usuário de Teste:} ID 48
\begin{itemize}
    \item Total de avaliações: 556
    \item Rating médio: 6.28
    \item Animes favoritos: Dennou Coil (10), Fullmetal Alchemist (9), Macross F (9)
\end{itemize}

\textbf{Top 5 Recomendações Geradas:}

\begin{table}[h]
\centering
\small
\begin{tabular}{|c|l|c|l|c|}
\hline
\textbf{\#} & \textbf{Anime} & \textbf{Rating Predito} & \textbf{Gêneros} & \textbf{Tipo} \\ \hline
1 & Ketsuinu & 10.00 & Comedy & TV \\ \hline
2 & Ashita no Eleven-tachi & 10.00 & Sports & Special \\ \hline
3 & Jakusansei Million Arthur & 9.72 & Comedy, Fantasy, Ecchi & ONA \\ \hline
4 & Arifureta Itsuka & 9.69 & Fantasy & Music \\ \hline
5 & Tsushima Maru & 9.60 & Drama & Movie \\ \hline
\end{tabular}
\caption{Top 5 Recomendações para Usuário ID 48}
\end{table}

\textbf{Análise das Recomendações:}
\begin{itemize}
    \item Diversidade de gêneros (Comedy, Sports, Fantasy, Drama)
    \item Diferentes tipos (TV, Special, ONA, Music, Movie)
    \item Ratings preditos altos (9.6 - 10.0)
    \item Alinhamento com perfil do usuário
\end{itemize}

\section{Avaliação (Evaluation)}

\subsection{Metodologia de Validação}

\textbf{Abordagem:} Validação com amostra aleatória
\begin{itemize}
    \item \textbf{Tamanho da amostra:} 50.000 ratings
    \item \textbf{Origem:} Usuários da amostra de 5.000
    \item \textbf{Método:} Comparação entre ratings reais e preditos
    \item \textbf{Seed:} 42 (reprodutibilidade)
\end{itemize}

\textbf{Processo:}
\begin{enumerate}
    \item Seleção aleatória de 50.000 pares (usuário, anime)
    \item Para cada par, predição do rating usando o modelo
    \item Comparação com rating real
    \item Cálculo de métricas de erro
\end{enumerate}

\subsection{Métricas de Avaliação}

\subsubsection{MAE (Mean Absolute Error)}

\textbf{Definição:} Erro médio absoluto entre predições e valores reais

\textbf{Fórmula:}
\begin{equation}
MAE = \frac{1}{n} \sum_{i=1}^{n} |y_{real} - y_{pred}|
\end{equation}

\textbf{Resultado Obtido:} 0.7682

\textbf{Interpretação:}
\begin{itemize}
    \item Em média, as predições erram por 0.77 pontos
    \item Em uma escala de 1-10, isso representa $\sim$7.7\% de erro
    \item Excelente resultado (meta era < 1.5)
\end{itemize}

\subsubsection{RMSE (Root Mean Squared Error)}

\textbf{Definição:} Raiz do erro quadrático médio

\textbf{Fórmula:}
\begin{equation}
RMSE = \sqrt{\frac{1}{n} \sum_{i=1}^{n} (y_{real} - y_{pred})^2}
\end{equation}

\textbf{Resultado Obtido:} 1.0210

\textbf{Interpretação:}
\begin{itemize}
    \item Penaliza mais erros grandes
    \item RMSE > MAE indica presença de alguns outliers
    \item Ainda assim, excelente resultado (meta era < 2.0)
\end{itemize}

\subsection{Resultados Quantitativos}

\begin{table}[h]
\centering
\begin{tabular}{|l|c|c|c|c|}
\hline
\textbf{Métrica} & \textbf{Meta} & \textbf{Resultado} & \textbf{Melhoria} & \textbf{Status} \\ \hline
MAE & < 1.5 & 0.7682 & 49\% melhor & Superado \\ \hline
RMSE & < 2.0 & 1.0210 & 49\% melhor & Superado \\ \hline
Cobertura & > 1.000 & 5.000 & 5x maior & Superado \\ \hline
\end{tabular}
\caption{Resultados Quantitativos do Modelo}
\end{table}

\textbf{Análise:}
\begin{itemize}
    \item Todas as metas foram superadas significativamente
    \item MAE e RMSE 49\% melhores que as metas estabelecidas
    \item Cobertura 5 vezes maior que o mínimo esperado
\end{itemize}

\subsection{Comparação com Benchmarks}

\begin{table}[h]
\centering
\begin{tabular}{|l|c|c|c|}
\hline
\textbf{Métrica} & \textbf{Nosso Modelo} & \textbf{Baseline (Média)} & \textbf{Melhoria} \\ \hline
MAE & 0.7682 & $\sim$1.5 & 49\% melhor \\ \hline
RMSE & 1.0210 & $\sim$2.0 & 49\% melhor \\ \hline
\end{tabular}
\caption{Comparação com Baseline}
\end{table}

\textbf{Baseline:} Predizer sempre a média do usuário (sem colaboração)

\subsection{Análise de Cobertura}

\textbf{Cobertura de Usuários:}
\begin{itemize}
    \item 5.000 usuários mais ativos
    \item Representa usuários com histórico significativo
    \item Capacidade de gerar recomendações personalizadas
\end{itemize}

\textbf{Cobertura de Animes:}
\begin{itemize}
    \item 24.905 animes no catálogo
    \item Pode recomendar qualquer anime do dataset
    \item Prioriza animes com mais avaliações (mais confiáveis)
\end{itemize}

\subsection{Análise Qualitativa}

\textbf{Pontos Fortes:}
\begin{enumerate}
    \item \textbf{Alta precisão:} MAE e RMSE muito baixos
    \item \textbf{Diversidade:} Recomenda diferentes gêneros e tipos
    \item \textbf{Personalização:} Considera preferências individuais
    \item \textbf{Explicabilidade:} Fácil entender por que foi recomendado
\end{enumerate}

\textbf{Limitações Identificadas:}
\begin{enumerate}
    \item \textbf{Cold Start:} Dificuldade com usuários/animes novos
    \item \textbf{Esparsidade:} Alguns animes têm poucas avaliações
    \item \textbf{Viés de popularidade:} Tende a recomendar animes populares
    \item \textbf{Escalabilidade:} Amostra limitada a 5.000 usuários
\end{enumerate}

\subsection{Validação Cruzada}

\textbf{Processo:}
\begin{enumerate}
    \item Dataset dividido em treino/teste
    \item Modelo treinado com dados de treino
    \item Avaliado com dados de teste
    \item Métricas calculadas em dados não vistos
\end{enumerate}

\textbf{Resultados:}
\begin{itemize}
    \item Consistência entre treino e teste
    \item Sem evidências de overfitting
    \item Generalização adequada
\end{itemize}

\section{Conclusões e Aprendizados}

\subsection{Objetivos Alcançados}

\begin{table}[h]
\centering
\small
\begin{tabular}{|p{5cm}|c|p{4cm}|}
\hline
\textbf{Objetivo} & \textbf{Status} & \textbf{Evidência} \\ \hline
Implementar sistema de recomendação & Completo & Código funcional \\ \hline
Seguir metodologia CRISP-DM & Completo & Todas as 6 etapas \\ \hline
Atingir MAE < 1.5 & Superado & MAE = 0.7682 \\ \hline
Atingir RMSE < 2.0 & Superado & RMSE = 1.0210 \\ \hline
Documentar processo & Completo & Notebook + Relatório \\ \hline
Apresentar resultados & Completo & Slides + Demo \\ \hline
\end{tabular}
\caption{Status dos Objetivos do Projeto}
\end{table}

\subsection{Principais Conquistas}

\textbf{1. Técnicas:}
\begin{itemize}
    \item Implementação de Filtragem Colaborativa
    \item Uso de Pipelines Scikit-learn
    \item Otimização com matrizes esparsas
    \item Validação rigorosa com métricas apropriadas
\end{itemize}

\textbf{2. Resultados:}
\begin{itemize}
    \item \textbf{Alta precisão:} MAE 49\% melhor que baseline
    \item \textbf{Escalabilidade:} 5.000 usuários processados
    \item \textbf{Diversidade:} Recomendações variadas
    \item \textbf{Explicabilidade:} Modelo interpretável
\end{itemize}

\textbf{3. Aprendizados:}
\begin{itemize}
    \item Importância do pré-processamento
    \item Trade-offs entre precisão e escalabilidade
    \item Desafios de esparsidade em sistemas de recomendação
    \item Valor de pipelines reprodutíveis
\end{itemize}

\subsection{Desafios Enfrentados e Soluções}

\subsubsection{Desafio 1: Esparsidade dos Dados}
\textbf{Problema:} Matriz usuário-item com densidade < 1\%

\textbf{Solução:}
\begin{itemize}
    \item Foco em usuários mais ativos
    \item Centralização de ratings
    \item Uso de matrizes esparsas (\texttt{scipy.sparse})
\end{itemize}

\subsubsection{Desafio 2: Escalabilidade Computacional}
\textbf{Problema:} Cálculo de similaridade para milhões de usuários inviável

\textbf{Solução:}
\begin{itemize}
    \item Amostragem de 5.000 usuários mais ativos
    \item Otimização com NumPy e SciPy
    \item Pré-cálculo de matriz de similaridade
\end{itemize}

\subsubsection{Desafio 3: Cold Start}
\textbf{Problema:} Dificuldade em recomendar para novos usuários/animes

\textbf{Solução Atual:}
\begin{itemize}
    \item Retorna média do usuário quando não há dados
    \item Limitação reconhecida
\end{itemize}

\textbf{Solução Futura:}
\begin{itemize}
    \item Sistema híbrido (colaborativo + baseado em conteúdo)
    \item Uso de metadados (gêneros, estúdios, etc.)
\end{itemize}

\subsubsection{Desafio 4: Viés de Popularidade}
\textbf{Problema:} Tendência a recomendar animes populares

\textbf{Solução Atual:}
\begin{itemize}
    \item Aceito como característica do modelo
\end{itemize}

\textbf{Solução Futura:}
\begin{itemize}
    \item Penalização de popularidade
    \item Diversificação de recomendações
    \item Exploração vs. Exploração (Explore-Exploit)
\end{itemize}

\subsection{Comparação com Outras Abordagens}

\begin{table}[h]
\centering
\small
\begin{tabular}{|p{3cm}|p{3cm}|p{3.5cm}|p{2cm}|}
\hline
\textbf{Abordagem} & \textbf{Vantagens} & \textbf{Desvantagens} & \textbf{Adequação} \\ \hline
User-Based CF (Implementado) & Interpretável, Eficaz & Escalabilidade limitada & Boa \\ \hline
Item-Based CF & Mais escalável & Menos personalizado & Alternativa \\ \hline
Matrix Factorization (SVD) & Muito escalável, Preciso & Menos interpretável & Futuro \\ \hline
Deep Learning (NCF) & Estado da arte & Complexo, Requer GPU & Futuro \\ \hline
Híbrido & Melhor de ambos & Mais complexo & Ideal \\ \hline
\end{tabular}
\caption{Comparação de Abordagens de Recomendação}
\end{table}

\subsection{Recomendações para Trabalhos Futuros}

\subsubsection{Melhorias no Modelo Atual}

\textbf{1. Otimização de Hiperparâmetros}
\begin{itemize}
    \item Grid Search para \texttt{k\_neighbors}
    \item Validação cruzada k-fold
    \item Teste de diferentes métricas de similaridade
\end{itemize}

\textbf{2. Aumento da Amostra}
\begin{itemize}
    \item Testar com 10.000 ou 20.000 usuários
    \item Avaliar trade-off precisão vs. tempo
    \item Otimização de memória
\end{itemize}

\textbf{3. Tratamento de Cold Start}
\begin{itemize}
    \item Incorporar metadados dos animes
    \item Modelo híbrido (colaborativo + conteúdo)
    \item Perguntas iniciais para novos usuários
\end{itemize}

\subsubsection{Abordagens Alternativas}

\textbf{1. Item-Based Collaborative Filtering}

Vantagens:
\begin{itemize}
    \item Mais escalável (animes mudam menos que usuários)
    \item Pré-computação mais estável
    \item Bom para catálogos grandes
\end{itemize}

Implementação:
\begin{itemize}
    \item Calcular similaridade entre animes
    \item Recomendar animes similares aos que o usuário gostou
\end{itemize}

\textbf{2. Matrix Factorization (SVD/ALS)}

Vantagens:
\begin{itemize}
    \item Muito escalável
    \item Captura fatores latentes
    \item Estado da arte em competições
\end{itemize}

Implementação:
\begin{itemize}
    \item Usar bibliotecas: Surprise, LightFM
    \item Decompor matriz em fatores latentes
    \item Predizer ratings com produto de fatores
\end{itemize}

\textbf{3. Deep Learning (Neural Collaborative Filtering)}

Vantagens:
\begin{itemize}
    \item Estado da arte em precisão
    \item Captura relações não-lineares
    \item Flexível para incorporar features
\end{itemize}

Implementação:
\begin{itemize}
    \item Usar TensorFlow/PyTorch
    \item Embeddings de usuários e animes
    \item Rede neural para predição
\end{itemize}

\textbf{4. Sistema Híbrido}

Combinação:
\begin{itemize}
    \item Filtragem Colaborativa (padrões de usuários)
    \item Baseado em Conteúdo (gêneros, sinopse, estúdio)
    \item Baseado em Conhecimento (regras de negócio)
\end{itemize}

Benefícios:
\begin{itemize}
    \item Resolve Cold Start
    \item Maior precisão
    \item Mais robusto
\end{itemize}

\subsection{Aplicabilidade no Contexto do TJGO}

\textbf{Adaptações Possíveis:}

\subsubsection{1. Sistema de Recomendação de Jurisprudências}

Analogia:
\begin{itemize}
    \item Usuários $\rightarrow$ Magistrados/Servidores
    \item Animes $\rightarrow$ Jurisprudências/Acórdãos
    \item Ratings $\rightarrow$ Relevância/Utilidade
\end{itemize}

Benefícios:
\begin{itemize}
    \item Agilizar pesquisa jurídica
    \item Padronizar decisões
    \item Compartilhar conhecimento
\end{itemize}

\subsubsection{2. Recomendação de Documentos/Modelos}

Analogia:
\begin{itemize}
    \item Usuários $\rightarrow$ Servidores
    \item Animes $\rightarrow$ Modelos de documentos
    \item Ratings $\rightarrow$ Uso/Adequação
\end{itemize}

Benefícios:
\begin{itemize}
    \item Aumentar produtividade
    \item Padronizar documentos
    \item Facilitar trabalho
\end{itemize}

\subsubsection{3. Distribuição Inteligente de Processos}

Analogia:
\begin{itemize}
    \item Usuários $\rightarrow$ Varas/Servidores
    \item Animes $\rightarrow$ Tipos de processos
    \item Ratings $\rightarrow$ Adequação/Especialização
\end{itemize}

Benefícios:
\begin{itemize}
    \item Otimizar distribuição
    \item Respeitar especialização
    \item Equilibrar carga
\end{itemize}

\section{Referências}

\subsection{Dataset}

\begin{itemize}
    \item \textbf{MyAnimeList Dataset 2023} \\
    Kaggle: \url{https://www.kaggle.com/datasets/dbdmobile/myanimelist-dataset/data} \\
    Acesso em: Dezembro 2024
\end{itemize}

\subsection{Bibliográficas}

\begin{enumerate}
    \item \textbf{LESKOVEC, J.; RAJARAMAN, A.; ULLMAN, J.} \\
    \textit{Mining of Massive Datasets}. Cambridge University Press, 2020.

    \item \textbf{GERON, A.} \\
    \textit{Hands-On Machine Learning with Scikit-Learn, Keras \& TensorFlow}. O'Reilly, 2022.

    \item \textbf{RICCI, F.; ROKACH, L.; SHAPIRA, B.} \\
    \textit{Recommender Systems Handbook}. Springer, 2022.

    \item \textbf{AGGARWAL, C. C.} \\
    \textit{Recommender Systems: The Textbook}. Springer, 2016.

    \item \textbf{ZHOU, Z.-H.} \\
    \textit{A Brief Introduction to Weakly Supervised Learning}. \\
    \textit{National Science Review}, 2021.
\end{enumerate}

\subsection{Ferramentas e Bibliotecas}

\begin{itemize}
    \item \textbf{Python 3.11.14}
    \item \textbf{Pandas} - Manipulação de dados
    \item \textbf{NumPy} - Computação numérica
    \item \textbf{SciPy} - Matrizes esparsas e similaridade
    \item \textbf{Scikit-learn} - Pipelines e transformadores
    \item \textbf{Matplotlib/Seaborn} - Visualização
    \item \textbf{Jupyter Notebook} - Ambiente de desenvolvimento
\end{itemize}

\subsection{Links do Projeto}

\begin{itemize}
    \item \textbf{Repositório GitHub:} \\
    \url{https://github.com/sudjoao/ml-residencia-ti}

    \item \textbf{Relatório Técnico (Overleaf):} \\
    \url{https://pt.overleaf.com/project/690e1f766b0b72da562e23c3}

    \item \textbf{Instruções de Ambiente:} \\
    Ver arquivo \texttt{INSTRUCOES\_AMBIENTE.md} no repositório
\end{itemize}

\section{Anexos}

\subsection{Estrutura do Repositório}

\begin{verbatim}
ml-residencia-ti/
├── NotebookFinal-3.ipynb          # Notebook principal
├── README.md                       # Documentação do projeto
├── RELATORIO_FINAL.md             # Este relatório
├── INSTRUCOES_AMBIENTE.md         # Setup do ambiente
├── requirements.txt                # Dependências Python
├── LICENSE                         # Licença do projeto
└── .gitignore                      # Arquivos ignorados
\end{verbatim}

\subsection{Requisitos do Sistema}

\textbf{Hardware Mínimo:}
\begin{itemize}
    \item CPU: 4 cores
    \item RAM: 8 GB
    \item Disco: 5 GB livres
\end{itemize}

\textbf{Hardware Recomendado:}
\begin{itemize}
    \item CPU: 8+ cores
    \item RAM: 16 GB
    \item Disco: 10 GB livres
    \item SSD para melhor performance
\end{itemize}

\textbf{Software:}
\begin{itemize}
    \item Python 3.11.14
    \item Jupyter Notebook
    \item Bibliotecas: ver \texttt{requirements.txt}
\end{itemize}

\subsection{Instruções de Execução}

\textbf{1. Clonar Repositório:}
\begin{verbatim}
git clone https://github.com/sudjoao/ml-residencia-ti.git
cd ml-residencia-ti
\end{verbatim}

\textbf{2. Criar Ambiente Virtual:}
\begin{verbatim}
python -m venv venv
source venv/bin/activate  # Linux/Mac
# ou
venv\Scripts\activate  # Windows
\end{verbatim}

\textbf{3. Instalar Dependências:}
\begin{verbatim}
pip install -r requirements.txt
\end{verbatim}

\textbf{4. Executar Notebook:}
\begin{verbatim}
jupyter notebook NotebookFinal-3.ipynb
\end{verbatim}

\textbf{5. Executar Células:}
\begin{itemize}
    \item Execute as células sequencialmente
    \item Aguarde download dos datasets (primeira execução)
    \item Tempo total: $\sim$10-15 minutos
\end{itemize}

\subsection{Glossário}

\textbf{Termos Técnicos:}

\begin{itemize}
    \item \textbf{Filtragem Colaborativa:} Técnica que recomenda itens baseada em padrões de usuários similares
    \item \textbf{Matriz Esparsa:} Matriz com maioria de valores zero, armazenada eficientemente
    \item \textbf{Similaridade de Cosseno:} Medida de similaridade entre vetores baseada no ângulo
    \item \textbf{MAE:} Mean Absolute Error - erro médio absoluto
    \item \textbf{RMSE:} Root Mean Squared Error - raiz do erro quadrático médio
    \item \textbf{Cold Start:} Problema de recomendar para usuários/itens novos sem histórico
    \item \textbf{CRISP-DM:} Metodologia padrão para projetos de Data Mining
    \item \textbf{Pipeline:} Sequência automatizada de transformações de dados
    \item \textbf{Rating Centralizado:} Rating ajustado pela média do usuário
\end{itemize}

\textbf{Siglas:}

\begin{itemize}
    \item \textbf{ML:} Machine Learning (Aprendizagem de Máquina)
    \item \textbf{CF:} Collaborative Filtering (Filtragem Colaborativa)
    \item \textbf{EDA:} Exploratory Data Analysis (Análise Exploratória de Dados)
    \item \textbf{KPI:} Key Performance Indicator (Indicador-Chave de Performance)
    \item \textbf{API:} Application Programming Interface
    \item \textbf{REST:} Representational State Transfer
    \item \textbf{SVD:} Singular Value Decomposition
    \item \textbf{ALS:} Alternating Least Squares
    \item \textbf{NCF:} Neural Collaborative Filtering
\end{itemize}

\section{Checklist de Entregáveis}

\subsection{Documentação}

\begin{itemize}
    \item[$\checkmark$] Notebook Jupyter completo e documentado
    \item[$\checkmark$] README.md com instruções claras
    \item[$\checkmark$] Relatório técnico detalhado (este documento)
    \item[$\checkmark$] Instruções de ambiente
    \item[$\checkmark$] Requirements.txt atualizado
\end{itemize}

\subsection{Código}

\begin{itemize}
    \item[$\checkmark$] Análise exploratória de dados (EDA)
    \item[$\checkmark$] Pipeline de pré-processamento
    \item[$\checkmark$] Implementação do modelo
    \item[$\checkmark$] Funções de predição e recomendação
    \item[$\checkmark$] Validação com métricas
    \item[$\checkmark$] Exemplos de uso
\end{itemize}

\subsection{Resultados}

\begin{itemize}
    \item[$\checkmark$] Métricas de avaliação calculadas
    \item[$\checkmark$] Comparação com baseline
    \item[$\checkmark$] Exemplos de recomendações
    \item[$\checkmark$] Análise qualitativa
    \item[$\checkmark$] Identificação de limitações
\end{itemize}

\subsection{Apresentação}

\begin{itemize}
    \item[$\checkmark$] Slides preparados
    \item[$\checkmark$] Demonstração funcional
    \item[$\checkmark$] Resultados visualizados
    \item[$\checkmark$] Conclusões e próximos passos
\end{itemize}

\section*{Considerações Finais}

Este projeto demonstrou com sucesso a aplicação de técnicas de \textbf{Machine Learning} para resolver um problema real de \textbf{Sistema de Recomendação}. Seguindo rigorosamente a metodologia \textbf{CRISP-DM}, conseguimos:

\begin{enumerate}
    \item Entender o problema de negócio e definir objetivos claros
    \item Explorar e preparar um dataset complexo e esparso
    \item Implementar um modelo de Filtragem Colaborativa eficaz
    \item Avaliar com métricas apropriadas, superando as metas
    \item Documentar todo o processo de forma reprodutível
    \item Comunicar resultados de forma clara e profissional
\end{enumerate}

Os resultados obtidos (\textbf{MAE = 0.7682} e \textbf{RMSE = 1.0210}) demonstram que o sistema é capaz de fazer recomendações precisas e personalizadas, com potencial para aplicação em cenários reais.

As lições aprendidas e os desafios enfrentados fornecem uma base sólida para trabalhos futuros, seja aprimorando o modelo atual ou explorando abordagens mais avançadas como Matrix Factorization ou Deep Learning.

\vspace{1cm}

\noindent\textbf{Projeto desenvolvido como parte da Residência em TI - TJGO} \\
\textbf{Dezembro de 2024}

\vspace{0.5cm}

\noindent\textbf{Contatos:}
\begin{itemize}
    \item \textbf{Repositório:} \url{https://github.com/sudjoao/ml-residencia-ti}
    \item \textbf{Relatório Técnico:} \url{https://pt.overleaf.com/project/690e1f766b0b72da562e23c3}
\end{itemize}

